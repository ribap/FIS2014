%%

NEXT (Neutrino Experiment with a Xenon TPC) es un experimento para buscar desintegraciones doble beta sin neutrinos (\bbonu), cuya detección demostraría unívocamente que el neutrino es una partícula de Majorana (es decir su propia antipartícula) y supondría un descubrimiento con profundas consecuencias en física de partículas y cosmología. 

 El isótopo escogido por NEXT es el \XE. El experimento dispone de cien kilos de gas xenón enriquecido al 90\% en \XE. La tecnología se basa en el uso de cámaras de proyección temporal operando a una presión típica de 15 atmósferas (\HPXE). Las características principales de esta técnica experimental son: a) excelente resolución en la medida de la energía; b) capacidad de reconstruir la trayectoria de los electrones emitidos en la desintegración, lo que refuerza la capacidad del experimento para reducir el ruido de fondo; c) escalabilidad a grandes masa y d) la posibilidad de reducir el ruido de fondo hasta niveles despreciables mediante la técnica conocida como \BATA\ (de las siglas en inglés, {\em Barium Tagging}).

 El experimento NEXT contempla cuatro fases: i) Demostración de la tecnología \HPXE\ con prototipos que usan $\sim$1 kg de xenón natural; ii) Medida de los ruidos de fondo y de la señal del proceso permitido (\bbtnu) con un detector NEW) basado en 12 kilos de xenón enriquecido y operando en el Laboratorio Subterráneo de Canfranc (LSC); iii) Búsqueda de desintegraciones \bbonu\ con el detector NEXT-100, una réplica a escala 2:1 (en tamaño) y 8:1 (en masa) de NEW, que usará por tanto 100 kilos de gas enriquecido; iv) Búsqueda de desintegraciones \bbonu\ con el detector BEXT (Barium-tagging Experiment with a Xenon TPC), con una masa de alrededor de una tonelada de \XE, que introducirá la técnica de \BATA\ para reducir el ruido de fondo hasta niveles despreciables. 

La primera fase de NEXT ha sido completada con éxito durante el periodo 2009-2013. Durante esta etapa se han construido los prototipos NEXT-DEMO (IFIC) y NEXT-DBDM (Berkeley) que han demostrado las características principales de la tecnología. El experimento se encuentra en estos momentos en su segunda fase. El detector NEW está siendo construido por la colaboración y operará en el LSC durante el año 2015. La financiación del detector NEW proviene de un Advanced Grant (AdG/ERC) concedido en 2013 al IP de este proyecto (operativo desde Febrero de 2014 a Febrero de 2018). El detector NEXT-100 supone la tercera fase del proyecto. Se construirá y pondrá a punto durante 2016 y 2017 e iniciará su toma de datos en 2018. La cuarta fase depende de los resultados de la fase tres, en la que se podría realizar ya un descubrimiento. Previsiblemente, BEXT podría funcionar en el LSC a partir del 2020. 

 NEXT es  una colaboración internacional, liderada por grupos españoles y con una fuerte contribución de grupos norteamericanos. El desarrollo de la tecnología laser necesaria para el BaTa se realiza en colaboración con el Centro de Láseres Pulsados de Salamanca (CLPU). 

 Este  proyecto de investigación requiere {\em cofinanciación} para desarrollar la fase tres del experimento. Concretamente, se requiere: a) fondos para adquirir una parte de los equipos y material fungible necesarios para la construcción del detector NEXT-100 (cofinanciado por el AdG y los fondos provenientes de la colaboración); b) fondos para una parte del personal científico y técnico; y c) fondos para cofinanciar el R\&D dedicado al \BATA.  
